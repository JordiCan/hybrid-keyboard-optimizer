%%%%%%%%%%%%%%%%%%%%%%%%%%%%%%%%%%%%%%%%%%%%%%%%%%%%%%%%%%%%%%%%%%%%%%%%%%%%%%%
%                       CLASE DE DOCUMENTO Y PAQUETES                         %
%%%%%%%%%%%%%%%%%%%%%%%%%%%%%%%%%%%%%%%%%%%%%%%%%%%%%%%%%%%%%%%%%%%%%%%%%%%%%%%

\documentclass[11pt,spanish]{report}


\usepackage[utf8]{inputenc} 


\usepackage{graphicx} 

\usepackage[spanish]{babel}
\usepackage{amsmath} 
\usepackage{booktabs} 
\usepackage{multirow}
\usepackage{float}
\usepackage{tikz}
\usetikzlibrary{positioning, arrows.meta, calc, backgrounds, fit, shapes.geometric, matrix, chains}
\usepackage{pgfplots}
\pgfplotsset{compat=1.18}
\usepgfplotslibrary{groupplots}
\usepackage{natbib}            
\usepackage{pdfpages}
\usepackage{parskip}
\setlength{\parindent}{0pt}
\setlength{\parskip}{1em} 

%%%%%%%%%%%%%%%%%%%%%%%%%%%%%%%%%%%%%%%%%%%%%%%%%%%%%%%%%%%%%%%%%%%%%%%%%%%%%%%
%                 INFORMACIÓN PARA LA PORTADA Y EL DOCUMENTO                  %
%          (Rellena o modifica los campos según sea necesario)                %
%%%%%%%%%%%%%%%%%%%%%%%%%%%%%%%%%%%%%%%%%%%%%%%%%%%%%%%%%%%%%%%%%%%%%%%%%%%%%%%

\newcommand{\theuniversity}{Universitat Politècnica de València}
\newcommand{\theschool}{DSIC - Departament de Sistemes Informàtics i Computació}
\newcommand{\thedegree}{MIARFID - Màster en Intel·ligència Artificial y Reconocimiento de Formas}
\newcommand{\theprojecttype}{Trabajo académico}

\newcommand{\thetitle}{Optimización de Distribución de Teclados para el Inglés}
\newcommand{\theauthor}{Jordi Cantavella Ferrero}
\newcommand{\thecourse}{Curs Acadèmic 2025-2026}


%%%%%%%%%%%%%%%%%%%%%%%%%%%%%%%%%%%%%%%%%%%%%%%%%%%%%%%%%%%%%%%%%%%%%%%%%%%%%%%
%                              INICIO DEL DOCUMENTO                           %
%%%%%%%%%%%%%%%%%%%%%%%%%%%%%%%%%%%%%%%%%%%%%%%%%%%%%%%%%%%%%%%%%%%%%%%%%%%%%%%
\usepackage[colorlinks=true, linkcolor=purple, citecolor=green!60!black, urlcolor=magenta]{hyperref}

\begin{document}

\begin{titlepage}
    \centering
    
   
    \vspace{1.5cm}
    
    {\Large \bfseries \theuniversity} \\
    \vspace{0.5cm}
    {\large \theschool} \\
    
    \vspace{2cm}
    
    {\large \textbf{\theprojecttype}} \\
    \vspace{0.2cm}
    {\large \thedegree} \\
    
    \vfill 
    
    {\Huge \bfseries \thetitle} \\
    
    \vfill
    
    \begin{minipage}{0.8\textwidth}
        \begin{flushleft}
            \large
            \textbf{Autor:} \\
            \theauthor \\
            \vspace{1cm}
        \end{flushleft}
    \end{minipage}
    
    \vfill
    
    {\large \thecourse}
    
\end{titlepage}


\tableofcontents
\listoffigures
\listoftables
\cleardoublepage

%%%%%%%%%%%%%%%%%%%%%%%%%%%%%%%%%%%%%%%%%%%%%%%%%%%%%%%%%%%%%%%%%%%%%%%%%%%%%%%
%                    ESTRUCTURA PRINCIPAL DEL CONTENIDO                       %
%%%%%%%%%%%%%%%%%%%%%%%%%%%%%%%%%%%%%%%%%%%%%%%%%%%%%%%%%%%%%%%%%%%%%%%%%%%%%%%

\chapter{Introducción} \label{chpt:1}
\section{Contexto}
La distribución de los teclados actuales no estan optimizadas para la ergonomia y la eficiencia a la hora de escribir. 
Las distribuciones actuales tienen sus raices en la mecanografía tradicional y no han evolucionado para adaptarse a las necesidades modernas de los usuarios.
Distribuciones como QWERTY, surgida en 1873, diseñadas para maquinas de escribir mecánicas hoy en día no son las más eficientes para la escritura en dispositivos digitales.


Estudios posteriores demostraron que la disposición del teclado QWERTY generaba movimientos ineficientes, contribuyendo a la fatiga y al riesgo de lesiones.
Dando lugar a posteriores distribuciones como Dvorak (1936) y Colemak (2006) que buscaban mejorar la eficiencia y reducir la fatiga del usuario.
Buscando maximizar la velocidad de escritura y minimizar el esfuerzo físico, estas distribuciones se basan en principios ergonómicos y análisis lingüísticos.

\section{Objetivos}
El objetivo principal de este trabajo es la optimización de la distribución del teclado, buscando generar disposiciones de teclado ergonómicas y eficientes.
Implementado para ello un algoritmo genético y enfriamiento simulado que permita explorar el espacio de posibles distribuciones y encontrar configuraciones óptimas 
basadas en criterios ergonómicos y lingüísticos.
\begin{itemize}
    \item Implementar un algoritmo genético que permita la generación y evaluación de diferentes distribuciones de teclado.
    \item Implementar un algoritmo de enfriamiento simulado para el mismo problema.
    \item Definir métricas de evaluación que consideren distancia, alternancia de manos y esfuerzo de dedos
    \item Análisis de los resultados obtenidos por los algoritmos.
    \item Comparar las distribuciones generadas con las existentes como \textit{QWERTY}, \textit{Dvorak} y \textit{Colemak}.
\end{itemize}

\section{Alcance y limitaciones}
Dadas las limitaciones temporales y de recursos, este trabajo se centrará en la optimización de distribuciones de teclado para el idioma español.
No se abordarán aspectos relacionados con la implementación física de los teclados ni la adaptación a diferentes dispositivos.

\chapter{Marco teórico} \label{chpt:2}
\section{Historia de las dispociones de teclado} \label{sec:historia_teclados}
Al largo del tiempo se han utilizado diferentes tipos de teclados, desde los primeros teclados mecánicos hasta los modernos teclados digitales.
Entre ellos se han destacado los teclados QWERTY, Dvorak y Colemak.

El teclado \textbf{QWERTY}, diseñado en 1873 por Christopher Latham Sholes, fue creado para evitar el atasco de las teclas en las primeras máquinas de escribir mecánicas.
Su diseño priorizaba la separación de las letras más utilizadas en inglés para reducir la velocidad de escritura y evitar que las barras de las teclas se atascaran.

El teclado \textbf{Dvorak}, desarrollado en 1936 por August Dvorak y su cuñado William Dealey, se diseñó para mejorar la eficiencia y reducir la fatiga del usuario.
Demostro una reducción de hasta un 35\% de movimiento comparandola con la distribución QWERTY, colocando las letras más utilizadas en la fila central y promoviendo la alternancia de manos.

El teclado \textbf{Colemak}, introducido en 2006 por Shai Coleman, es una evolución del teclado QWERTY que busca mejorar la eficiencia sin requerir un cambio radical en la disposición de las teclas.
Colemak mantiene muchas de las teclas en sus posiciones originales, facilitando la transición para los usuarios acostumbrados a QWERTY.
Se enfoca en minimizar el movimiento de los dedos y maximizar la velocidad de escritura, colocando las letras más comunes en posiciones accesibles.

Como muestra la Figura \ref{fig:keyboard_layouts}, cada una de estas distribuciones tiene un enfoque diferente en cuanto a la ergonomía y la eficiencia, reflejando las necesidades y tecnologías de su época.

\begin{figure}[H]
    \centering
    \includegraphics[width=0.7\textwidth]{images/qwerty-dvorak-colemak.jpg}
    \caption{Comparación de las distribuciones de teclado QWERTY, Dvorak y Colemak.}
    \label{fig:keyboard_layouts}

\end{figure}

Representando las diferentes distribuciones de teclado, la Figura \ref{fig:keyboard_layouts} ilustra cómo cada una de estas disposiciones busca optimizar la experiencia de escritura desde diferentes perspectivas.
Observando la concentración de las letras más utilizadas en la fila central del teclado Dvorak y Colemak, en contraste con la disposición más dispersa del teclado QWERTY.

\section{Ergonomia del teclado} \label{sec:ergonomia_teclado}
Siguiendo con las principales distribuciones introducidas en la sección \ref{sec:historia_teclados}, la ergonomía del teclado es un aspecto crucial en el diseño de las disposiciones de teclado.
La mano presenta una serie de carácterísticas anatómicas que influyen en la comodidad y eficiencia al escribir. Entre ella destacamos:
\begin{itemize}
    \item \textbf{Fuerza de los dedos:} La fuerza de los dedos varía, siendo el pulgar y el dedo medio los más fuertes, mientras que el meñique es el más débil.
    \item \textbf{Movilidads de los dedos:} Los dedos tienen diferentes rangos de movimiento, con el pulgar y el dedo índice siendo los más móviles.
    \item \textbf{Fatiga muscular:} El uso prolongado de ciertos dedos puede llevar a la fatiga muscular, especialmente si se requiere un esfuerzo constante.
    \item \textbf{Postura de la mano:} Una postura natural y relajada de la mano es esencial para evitar tensiones y lesiones a largo plazo.
\end{itemize}

\section{Analisis linguistico} \label{sec:analisis_linguistico}
El análisis lingüístico es fundamental para diseñar distribuciones de teclado que optimicen la eficiencia de la escritura.
Prestando atención a la frecuencia de las letras y su distribución nos permitira diseñar teclados que minimicen el movimiento de los dedos y maximicen la velocidad de escritura.

En la Figura \ref{fig:english_letter_frequency} se muestra un análisis de la frecuencia de las letras en ingles.

\begin{figure}[H]
    \centering
    \includegraphics[width=0.8\textwidth]{images/english.png}
    \caption{Frecuencia de las letras en ingles.}
    \label{fig:english_letter_frequency}
\end{figure}

La Figura \ref{fig:letter_frequency_languages} muestra la frecuencia de distribución de las 26 letras más comunes utilizadas en el latín.
Siendo todas las lenguas que utilizan el alfabeto latino, como el español, inglés, francés, alemán e italiano.
Observando que la frecuencia de las letras varía entre los diferentes idiomas, lo que influye en el diseño óptimo de las distribuciones de teclado para cada lengua, 
aún siendo estas lenguas con un alfabeto similar y origen común.

\begin{figure}[H]
    \centering
    \includegraphics[width=1\textwidth]{images/idiomas.png}
    \caption{Frecuencia de las letras en diferentes idiomas.}
    \label{fig:letter_frequency_languages}
\end{figure}

Observado queda en la Figura \ref{fig:letter_frequency_languages} que la frecuencia entre diferentes idiomas varia.
También a destacar que nuestro corpus de referencia, ha de capturar un conjunto diverso del uso de la lengua, incluyendo textos formales e informales, para obtener una representación precisa de la frecuencia 
de las letras y combinaciones de letras.



%\section{Speaker diarization} \label{sec:speaker_diarization}

\chapter{Metodologia Algoritmo Genético} \label{chpt:3}
\section{Modelado del problema} \label{sec:modelado_problema}
\section{Función de evaluación} \label{sec:funcion_evaluacion}
\section{Representación} \label{sec:representacion}
\section{Diseño del Algoritmo Genético} \label{sec:diseno_algoritmo_genetico}

\chapter{Multiple speaker experiments in English} \label{chpt:4}
\section{Introduction} \label{Experimental Introduction} 
\section{Pipeline construction} \label{Pipeline construction}
\section{Experimental results} \label{sec:Experimental results English}
\section{Conclusion} \label{sec:Conclusion English}

\chapter{Multiple speaker experiments in Catalan} \label{chpt:5}
\section{Introduction} \label{sec:introduction_catalan}
\section{Programs} \label{sec:programs_catalan}
\section{Experimental setup} \label{sec:experimental_setup_catalan}
\section{Experiments results} \label{sec:experiments_results_catalan}
\section{Conclusion} \label{sec:conclusion_catalan}

\chapter{Conclusions} \label{chpt:6}
\section{Summary of the developed work}
\section{Achieved objectives}
\section{Future work}

%%%%%%%%%%%%%%%%%%%%%%%%%%%%%%%%%%%%%%%%%%%%%%%%%%%%%%%%%%%%%%%%%%%%%%%%%%%%%%%
%                                BIBLIOGRAFÍA                                 %
%%%%%%%%%%%%%%%%%%%%%%%%%%%%%%%%%%%%%%%%%%%%%%%%%%%%%%%%%%%%%%%%%%%%%%%%%%%%%%%

\cleardoublepage
% \bibliographystyle{plain}
% \bibliography{referencias} % Descomenta si tienes un archivo .bib

%%%%%%%%%%%%%%%%%%%%%%%%%%%%%%%%%%%%%%%%%%%%%%%%%%%%%%%%%%%%%%%%%%%%%%%%%%%%%%%
%                                 APÉNDICES                                   %
%%%%%%%%%%%%%%%%%%%%%%%%%%%%%%%%%%%%%%%%%%%%%%%%%%%%%%%%%%%%%%%%%%%%%%%%%%%%%%%

\appendix
\chapter{Sustainable Development Goals (ODS)}
% Contenido del apéndice aquí

%%%%%%%%%%%%%%%%%%%%%%%%%%%%%%%%%%%%%%%%%%%%%%%%%%%%%%%%%%%%%%%%%%%%%%%%%%%%%%%
%                              FIN DEL DOCUMENTO                              %
%%%%%%%%%%%%%%%%%%%%%%%%%%%%%%%%%%%%%%%%%%%%%%%%%%%%%%%%%%%%%%%%%%%%%%%%%%%%%%%

\end{document}