\documentclass[compress,aspectratio=169]{beamer}

% Paquetes básicos
\usepackage{tikz}
\usetikzlibrary{positioning, arrows.meta, shapes, matrix, shadows} 
% \usepackage[sfdefault]{roboto}  % Comentado - requiere instalación
\usepackage[utf8]{inputenc}
\usepackage[T1]{fontenc}
\usepackage{lmodern}  % Fuente alternativa Latin Modern
\usepackage{xcolor}
\usepackage{hyperref}
\usepackage{multimedia}
\usepackage{multirow}
\usepackage{tabularx,booktabs}
\usepackage[absolute,overlay]{textpos}
\usepackage{subcaption}
\usepackage[spanish,es-noshorthands]{babel}
\usepackage{pgfpages}
\usepackage{ragged2e}
\usepackage{textcomp}
\usepackage{pgfplots}
\usepackage{appendixnumberbeamer}
\usepackage{fontawesome5}
\usepackage{algorithm}
\usepackage{algorithmic}
\usepackage{listings}

\usetikzlibrary{chains}

\pgfplotsset{compat=1.17} 

% Configuración de colores UPV
\definecolor{UPVBlue}{RGB}{0,84,159}
\definecolor{UPVRed}{RGB}{204,0,51}
\definecolor{UPVGray}{RGB}{102,102,102}
\definecolor{UPVLightGray}{RGB}{230,230,230}
\definecolor{ForestGreen}{RGB}{34, 139, 34}

% Tema y colores
\usetheme[hideallsubsections,width=2.5cm]{Hannover}
\usecolortheme{dolphin}
\usecolortheme[named=UPVBlue]{structure}

% Configuraciones adicionales
\setbeamertemplate{navigation symbols}{}
\setbeameroption{hide notes}
\setbeamertemplate{note page}{\pagecolor{white}\large{\insertnote}}

\setbeamertemplate{frametitle}{
  \begin{beamercolorbox}[wd=\paperwidth, left, leftskip=0.3cm]{frametitle}
    \usebeamerfont{frametitle}\insertframetitle
  \end{beamercolorbox}
}

% Configuración de código Python
\lstset{
  language=Python,
  basicstyle=\ttfamily\small,
  keywordstyle=\color{blue},
  commentstyle=\color{green!60!black},
  stringstyle=\color{orange},
  showstringspaces=false,
  breaklines=true
}

% Información del documento
\title{Optimización Híbrida de Layouts de Teclado mediante Algoritmos Genéticos y Enfriamiento Simulado}
\author[Jordi Cantavella Ferrero]{Jordi Cantavella Ferrero \\[1em]
\footnotesize{
\begin{tabular}{c}
  \\
\end{tabular}
}}
\institute[UPV]{
    MIARFID\\
    Universitat Politècnica de València
}
\titlegraphic{
  \begin{minipage}{0.45\linewidth}\centering
    \includegraphics[width=0.5\linewidth]{recursos/logo-upv.pdf}
  \end{minipage}%
  \begin{minipage}{0.45\linewidth}\centering
  \end{minipage}
}
\date{\scriptsize{Valencia, [Fecha]}}

% Configuración de la página de título
\makeatletter
\setbeamertemplate{title page}{
  \vbox{}
  \vfill
  \begingroup
    \centering
    \vskip1em
    \begin{beamercolorbox}[sep=8pt,center]{title}
      \usebeamerfont{title}\inserttitle\par
      \ifx\insertsubtitle\@empty
      \else
        \vskip0.25em
        {\usebeamerfont{subtitle}\usebeamercolor[fg]{subtitle}\insertsubtitle\par}
      \fi
    \end{beamercolorbox}
    \vskip1em\par
    \begin{beamercolorbox}[sep=8pt,center]{author}
      \usebeamerfont{author}\insertauthor
    \end{beamercolorbox}
    \begin{beamercolorbox}[sep=8pt,center]{institute}
      \usebeamerfont{institute}\insertinstitute
    \end{beamercolorbox}
    \vspace{-1em}
    \begin{beamercolorbox}[sep=8pt,center]{date}
      \usebeamerfont{date}\insertdate
    \end{beamercolorbox}
    \vskip0.5em
    {\usebeamercolor[fg]{titlegraphic}\inserttitlegraphic\par}
  \endgroup
}
\makeatother

% Logo en las diapositivas
\pgfdeclareimage[height=0.95cm]{upv-logo}{recursos/logo-upv.pdf}
\logo{\pgfuseimage{upv-logo}}

% Índice al inicio de cada sección
\AtBeginSection[]
{
  \begin{frame}<beamer>{Índice}
    \tableofcontents[currentsection,hideothersubsections]
  \end{frame}
}

% Configuración de la barra lateral
\def\swidth{2.6cm}
\setbeamersize{sidebar width left=\swidth}

%%%%%%%%%%%%%%%%%%%%%%%%%%%%%%%%%%%%%%%%%%%%%%%%%%%%%%%%%%%%
% INICIO DEL DOCUMENTO
%%%%%%%%%%%%%%%%%%%%%%%%%%%%%%%%%%%%%%%%%%%%%%%%%%%%%%%%%%%%
\begin{document}

% --- PORTADA ---
\setbeamertemplate{sidebar left}{}
\begin{frame}[plain]
  \advance\textwidth-2.5cm
  \hsize\textwidth
  \columnwidth\textwidth
  \titlepage
\end{frame}

% --- ÍNDICE GENERAL ---
\setbeamertemplate{sidebar left}
{
      \hbox to 3cm{\insertlogo}
      \vskip1.25em%
      \insertverticalnavigation{\swidth}%
      \vfill
      \hbox to3cm{\hskip0.6cm\usebeamerfont{subsection in sidebar}
                  \strut\usebeamercolor[fg]{subsection in sidebar}
                  \insertframenumber/\inserttotalframenumber\hfill}%
      \vskip3pt%
}%
\begin{frame}{Índice}
  \tableofcontents[hideallsubsections]
\end{frame}


%% ============================================
%% SECCIÓN 1: INTRODUCCIÓN
%% ============================================
\section{Introducción}

\begin{frame}{El Problema del Diseño de Teclados}
  \begin{columns}
    % Columna izquierda - Bloques de texto
    \begin{column}{0.65\textwidth}
      \begin{block}{\underline{El Layout QWERTY}}
        \vspace{0.5em}
        Diseñado en 1873 para máquinas de escribir mecánicas, no para eficiencia ergonómica.
      \end{block}
      
      \vspace{1em}
      
      \begin{alertblock}{\textcolor{UPVRed}{\underline{Problemas Actuales}}}
        \vspace{0.5em}
        \begin{itemize}
          \item Alta distancia recorrida por los dedos
          \item Baja alternancia entre manos
          \item Distribución subóptima de teclas frecuentes
          \item Aumento de lesiones por esfuerzo repetitivo
        \end{itemize}
      \end{alertblock}

      \vspace{0.5em}

      \begin{beamercolorbox}[sep=8pt,left,rounded=true,shadow=true]{block body}
        \small\textit{¿Podemos encontrar un layout más eficiente}
      \end{beamercolorbox}

    \end{column}
    
    % Columna derecha - Visualización de teclado
    \begin{column}{0.35\textwidth}
      \centering
      \vspace{0.5em}
      
      % Representación simplificada de teclado QWERTY (3x10)
      \begin{tikzpicture}[scale=0.45, every node/.style={transform shape}]
        \tikzstyle{key} = [rectangle, draw=blue!60!black, fill=blue!15, 
          rounded corners=2pt, minimum width=0.8cm, minimum height=0.8cm, font=\small]
        
        % Primera fila
        \foreach \letter [count=\i] in {Q,W,E,R,T,Y,U,I,O,P}
          \node[key] at (\i*0.9, 3) {\letter};
        
        % Segunda fila
        \foreach \letter [count=\i] in {A,S,D,F,G,H,J,K,L,;}
          \node[key] at (\i*0.9, 2) {\letter};
        
        % Tercera fila
        \foreach \letter [count=\i] in {Z,X,C,V,B,N,M,{,},{.},{'}}
          \node[key] at (\i*0.9, 1) {\letter};
          
        % Etiqueta
        \node[font=\small\bfseries, color=UPVBlue] at (5, 0) {QWERTY};
      \end{tikzpicture}
      
      \vspace{1em}
      
      \Large\textcolor{UPVRed}{\faArrowDown}
      
      \vspace{0.5em}
      
      \begin{tikzpicture}[scale=0.45, every node/.style={transform shape}]
        \tikzstyle{key} = [rectangle, draw=green!60!black, fill=green!25, 
          rounded corners=2pt, minimum width=0.8cm, minimum height=0.8cm, font=\small]
        
        % Layout COLEMAK - 3x10
        \foreach \letter [count=\i] in {Q,W,F,P,G,J,L,U,Y,;}
          \node[key] at (\i*0.9, 3) {\letter};
        
        \foreach \letter [count=\i] in {A,R,S,T,D,H,N,E,I,O}
          \node[key] at (\i*0.9, 2) {\letter};
        
        \foreach \letter [count=\i] in {Z,X,C,V,B,K,M,{,},{.},{'}}
          \node[key] at (\i*0.9, 1) {\letter};
          
        % Etiqueta
        \node[font=\small\bfseries, color=ForestGreen] at (5, 0) {COLEMAK};
      \end{tikzpicture}
   
    \end{column}
  \end{columns}
\end{frame}

% \begin{frame}{Motivación del Trabajo}
%   \begin{block}{\underline{¿Por qué optimizar layouts de teclado?}}
%   \end{block}
  
%   \vspace{1em}
  
%   \begin{columns}[T]
%     % Columna 1: Salud
%     \begin{column}{0.33\textwidth}
%       \begin{beamercolorbox}[sep=8pt,center,rounded=true,shadow=true]{block body alerted}
%         \centering
%         \Large\faHeartbeat\\[0.5em]
%         \normalsize\textbf{Ergonomía}\\[0.5em]
%         \small Reducir lesiones por esfuerzo repetitivo (RSI)
%       \end{beamercolorbox}
%     \end{column}
    
%     % Columna 2: Eficiencia
%     \begin{column}{0.33\textwidth}
%       \begin{beamercolorbox}[sep=8pt,center,rounded=true,shadow=true]{block body example}
%         \centering
%         \Large\faTachometerAlt\\[0.5em]
%         \normalsize\textbf{Eficiencia}\\[0.5em]
%         \small Aumentar velocidad de escritura y reducir fatiga
%       \end{beamercolorbox}
%     \end{column}
    
%     % Columna 3: Investigación
%     \begin{column}{0.33\textwidth}
%       \begin{beamercolorbox}[sep=8pt,center,rounded=true,shadow=true]{block body successful}
%         \centering
%         \Large\faFlask\\[0.5em]
%         \normalsize\textbf{Investigación}\\[0.5em]
%         \small Aplicar metaheurísticas a problemas reales
%       \end{beamercolorbox}
%     \end{column}
%   \end{columns}
  
%   \vspace{1.5em}
  
%   \begin{alertblock}{\centering Estadística Relevante}
%     \centering
%     \Large El usuario promedio escribe \textbf{más de 1 millón de pulsaciones} al mes.\\
%     \small Pequeñas mejoras en eficiencia = gran impacto acumulado
%   \end{alertblock}
% \end{frame}

\begin{frame}{Objetivos del Trabajo}
  \begin{block}{\underline{Objetivo Principal}}
    \Large
    Conseguir mediante el uso de algoritmos genéticos y enfriamiento simulado, una distribución 
    óptima para un teclado.
  \end{block}
  
  \vspace{1.5em}
  
  \begin{center}
    \textbf{Objetivos Específicos:}
  \end{center}
  
  \vspace{1em}
  \centering
  \begin{tikzpicture}[scale=0.80, every node/.style={transform shape}, 
    node distance=0.8cm, 
    block/.style={rectangle, draw, rounded corners, text width=7.5em, 
      align=center, minimum height=4em, very thick},
    line/.style={draw, -{Latex[length=2.5mm]}, thick, color=UPVGray}]
    
    \node[block, draw=blue!70!black, fill=blue!20] (obj1) 
      {\textbf{1. Descripción del problema} \\};
    \node[block, draw=green!70!black, fill=green!20, right=of obj1] (obj2) 
      {\textbf{2. Algoritmo genético} \\ };
    \node[block, draw=yellow!70!black, fill=yellow!20, right=of obj2] (obj3) 
      {\textbf{3. Enfriamiento simulado} \\ };
    \node[block, draw=red!70!black, fill=red!20, right=of obj3] (obj4) 
      {\textbf{4. Comparativa} \\};
    
    \draw[line] (obj1) -- (obj2);
    \draw[line] (obj2) -- (obj3);
    \draw[line] (obj3) -- (obj4);
  \end{tikzpicture}
\end{frame}


%% ============================================
%% SECCIÓN 2: FUNDAMENTOS TEÓRICOS
%% ============================================
\section{Fundamentos Teóricos}

\begin{frame}{Función de Fitness: Visión General}
  \begin{block}{\underline{Evaluación Basada en Bigramas}}
    El fitness evalúa el costo de escribir pares de letras consecutivas según su frecuencia.
  \end{block}
  
  \vspace{1em}
  
  \begin{beamercolorbox}[sep=12pt,center,rounded=true,shadow=true]{block body}
    \Large
    $Fitness = \sum_{bigramas} (distancia \times multiplicador) \times frecuencia$
  \end{beamercolorbox}
  
  \vspace{1.5em}
  
  \begin{columns}[T]
    \begin{column}{0.5\textwidth}
      \begin{block}{Componentes}
        \begin{enumerate}
          \item Distancia euclidiana
          \item Penalizaciones por dedos
          \item Frecuencia del bigrama
        \end{enumerate}
      \end{block}
    \end{column}
    
    \begin{column}{0.5\textwidth}
      \begin{alertblock}{Objetivo}
        \centering
        \Large \textbf{MINIMIZAR}\\[0.5em]
        \normalsize el costo total
      \end{alertblock}
    \end{column}
  \end{columns}
  
  \vspace{1em}
  
  \begin{beamercolorbox}[sep=8pt,center,rounded=true,shadow=true]{block body successful}
    \small Se precomputan matrices 30×30 de distancias y penalizaciones
  \end{beamercolorbox}
\end{frame}

\begin{frame}{Métrica 1: Distancia Euclidiana}
  \begin{columns}[T]
    % Columna izquierda - Explicación
    \begin{column}{0.45\textwidth}
      \begin{block}{\underline{Distancia Física}}
        Distancia geométrica entre teclas en el espacio 2D.
      \end{block}
      
      \vspace{1em}
      
      \begin{beamercolorbox}[sep=8pt,center,rounded=true,shadow=true]{block body}
        $d = \sqrt{(\Delta fila)^2 + (\Delta col)^2}$
      \end{beamercolorbox}
      
      \vspace{1em}
      
      \begin{itemize}
        \item Layout 3×10 (posiciones 0-29)
        \item Menor distancia = más eficiente
      \end{itemize}
      

    \end{column}
    
    % Columna derecha - Visualización
    \begin{column}{0.55\textwidth}
      \centering
      
      % Distancia corta
      \begin{beamercolorbox}[sep=4pt,center]{block body successful}
        \textbf{Distancia CORTA}
      \end{beamercolorbox}
      \vspace{0.3em}
      \begin{tikzpicture}[scale=0.45, every node/.style={transform shape}]
        \tikzstyle{key} = [rectangle, draw=UPVGray!60!black, fill=UPVLightGray, 
          rounded corners=2pt, minimum width=0.8cm, minimum height=0.8cm, font=\small]
        \tikzstyle{highlight} = [rectangle, draw=ForestGreen!80!black, fill=green!40, 
          rounded corners=2pt, minimum width=0.8cm, minimum height=0.8cm, font=\small\bfseries]
        
        % Fila 1
        \foreach \letter [count=\i] in {Q,W,E,R,T,Y,U,I,O,P}
          \node[key] at (\i*0.9, 3) {\letter};

        % Fila 2 - destacar D y F
        \foreach \letter [count=\i] in {A,S}
          \node[key] at (\i*0.9, 2) {\letter};
        \node[highlight] at (2.7, 2) {D};
        \node[highlight] at (3.6, 2) {F};
        \foreach \letter [count=\i] in {G,H,J,K,L,;}
          \node[key] at (\i*0.9+3.6, 2) {\letter};
        
        % Fila 3
        \foreach \letter [count=\i] in {z,x,c,v,b,n,m,{,},{.},{'}}
          \node[key] at (\i*0.9, 1) {\letter};
          
        \draw[<->, thick, ForestGreen] (2.8, 2) -- (3.5, 2);
        \node[font=\large\bfseries, color=ForestGreen] at (3.15, 2.8) {$d = 1$};
      \end{tikzpicture}
      
      \vspace{1em}
      
      % Distancia larga
      \begin{beamercolorbox}[sep=4pt,center]{block body alerted}
        \textbf{Distancia LARGA}
      \end{beamercolorbox}
      \vspace{0.3em}
      \begin{tikzpicture}[scale=0.45, every node/.style={transform shape}]
        \tikzstyle{key} = [rectangle, draw=UPVGray!60!black, fill=UPVLightGray, 
          rounded corners=2pt, minimum width=0.8cm, minimum height=0.8cm, font=\small]
        \tikzstyle{bad} = [rectangle, draw=UPVRed!80!black, fill=UPVRed!40, 
          rounded corners=2pt, minimum width=0.8cm, minimum height=0.8cm, font=\small\bfseries]
        
        % Fila 1
        \node[bad] at (0.9, 3) {Q};
        \foreach \letter [count=\i] in {W,E,R,T,Y,U,I,O}
          \node[key] at (\i*0.9+0.9, 3) {\letter};
        \node[bad] at (9.0, 3) {P};
        
        % Fila 2
        \foreach \letter [count=\i] in {A,S,D,F,G,H,J,K,L,;}
          \node[key] at (\i*0.9, 2) {\letter};
        
        % Fila 3
        \foreach \letter [count=\i] in {Z,X,C,V,B,N,M,{,},{.},{'}}
          \node[key] at (\i*0.9, 1) {\letter};
          
        \draw[<->, ultra thick, UPVRed] (1.1, 3) -- (8.8, 3);
        \node[font=\large\bfseries, color=UPVRed] at (5, 3.9) {$d = 8.1$};
      \end{tikzpicture}
    \end{column}
  \end{columns}
\end{frame}

\begin{frame}{Métrica 2: Same-Finger Penalty}
  \begin{columns}[T]
    % Columna izquierda
    \begin{column}{0.45\textwidth}
      \begin{block}{\underline{Mismo Dedo}}
        Penalización cuando un bigrama usa el mismo dedo consecutivamente.
      \end{block}
      
      
      \begin{block}{Asignación}
        \small
        \begin{itemize}
          \item Cols 0,9: \textcolor{UPVRed}{Meñique} (str=1)
          \item Cols 1,8: \textcolor{orange}{Anular} (str=2)
          \item Cols 2,7: \textcolor{yellow}{Medio} (str=3)
          \item Cols 3-6: \textcolor{green}{Índice} (str=4)
        \end{itemize}
      \end{block}
      
      
      \begin{alertblock}{Penalización}
        \begin{itemize}
          \small
          \item Dedo fuerte: \textbf{+1.0}
          \item Dedo débil: \textbf{+2.0}
        \end{itemize}
      \end{alertblock}
    \end{column}
    
    % Columna derecha
    \begin{column}{0.55\textwidth}
      \centering
      
      \vspace{1.5em}
      % Mapa de dedos
      \begin{beamercolorbox}[sep=4pt,center]{block body example}
        \textbf{Asignación de Dedos}
      \end{beamercolorbox}
      \vspace{0.3em}
      \begin{tikzpicture}[scale=0.6, every node/.style={transform shape}]
        \tikzstyle{pinky} = [rectangle, draw=UPVRed!60!black, fill=UPVRed!25, 
          rounded corners=2pt, minimum width=0.8cm, minimum height=0.8cm, font=\small]
        \tikzstyle{ring} = [rectangle, draw=orange!60!black, fill=orange!25, 
          rounded corners=2pt, minimum width=0.8cm, minimum height=0.8cm, font=\small]
        \tikzstyle{middle} = [rectangle, draw=yellow!60!black, fill=yellow!25, 
          rounded corners=2pt, minimum width=0.8cm, minimum height=0.8cm, font=\small]
        \tikzstyle{index} = [rectangle, draw=ForestGreen!60!black, fill=green!25, 
          rounded corners=2pt, minimum width=0.8cm, minimum height=0.8cm, font=\small]
        
        % Primera fila completa (Q W E R T Y U I O P)
        \node[pinky] at (0.9, 3) {Q};
        \node[ring] at (1.8, 3) {W};
        \node[middle] at (2.7, 3) {E};
        \node[index] at (3.6, 3) {R};
        \node[index] at (4.5, 3) {T};
        \node[index] at (5.4, 3) {Y};
        \node[index] at (6.3, 3) {U};
        \node[middle] at (7.2, 3) {I};
        \node[ring] at (8.1, 3) {O};
        \node[pinky] at (9.0, 3) {P};
        
        % Segunda fila completa (A S D F G H J K L ;)
        \node[pinky] at (0.9, 2) {A};
        \node[ring] at (1.8, 2) {S};
        \node[middle] at (2.7, 2) {D};
        \node[index] at (3.6, 2) {F};
        \node[index] at (4.5, 2) {G};
        \node[index] at (5.4, 2) {H};
        \node[index] at (6.3, 2) {J};
        \node[middle] at (7.2, 2) {K};
        \node[ring] at (8.1, 2) {L};
        \node[pinky] at (9.0, 2) {;};
        
        % Tercera fila completa (Z X C V B N M , . ')
        \node[pinky] at (0.9, 1) {Z};
        \node[ring] at (1.8, 1) {X};
        \node[middle] at (2.7, 1) {C};
        \node[index] at (3.6, 1) {V};
        \node[index] at (4.5, 1) {B};
        \node[index] at (5.4, 1) {N};
        \node[index] at (6.3, 1) {M};
        \node[middle] at (7.2, 1) {,};
        \node[ring] at (8.1, 1) {.};
        \node[pinky] at (9.0, 1) {'};
      \end{tikzpicture}
      
      
      \vspace{2em}
      
      % Ejemplo
      \begin{beamercolorbox}[sep=4pt,center]{block body alerted}
        \textbf{Ejemplo: "ed" en QWERTY}
      \end{beamercolorbox}
      \vspace{0.3em}
      \begin{tikzpicture}[scale=0.6, every node/.style={transform shape}]
        \tikzstyle{key} = [rectangle, draw=UPVGray!60!black, fill=UPVLightGray, 
          rounded corners=2pt, minimum width=0.8cm, minimum height=0.8cm, font=\small]
        \tikzstyle{bad} = [rectangle, draw=UPVRed!80!black, fill=UPVRed!50, 
          rounded corners=2pt, minimum width=0.8cm, minimum height=0.8cm, font=\small\bfseries]
        
        % Fila 1
        \foreach \letter [count=\i] in {Q,W}
          \node[key] at (\i*0.9, 3) {\letter};
        \node[bad] at (2.7, 3) {E};
        \foreach \letter [count=\i] in {R,T,Y,U,I,O,P}
          \node[key] at (\i*0.9+2.7, 3) {\letter};
        
        % Fila 2
        \foreach \letter [count=\i] in {A,S}
          \node[key] at (\i*0.9, 2) {\letter};
        \node[bad] at (2.7, 2) {D};
        \foreach \letter [count=\i] in {F,G,H,J,K,L,;}
          \node[key] at (\i*0.9+2.7, 2) {\letter};
        
        % Fila 3
        \foreach \letter [count=\i] in {Z,X,C,V,B,N,M,{,},{.},{'}}
          \node[key] at (\i*0.9, 1) {\letter};
          
        \draw[<->, thick, UPVRed] (2.7, 2.8) -- (2.7, 2.2);
        \node[font=\large\bfseries, color=UPVRed] at (2.7, 3.7) {Dedo medio};
        \node[font=\large\bfseries, color=UPVRed] at (5, 0.2) {penalty +1.0};
      \end{tikzpicture}
    \end{column}
  \end{columns}
\end{frame}

\begin{frame}{Métrica 2: Same-Hand Penalty}
  \begin{columns}[T]
    % Columna izquierda
    \begin{column}{0.45\textwidth}
      \begin{block}{\underline{Misma Mano}}
        Penalización por usar dedos diferentes de la misma mano.
      \end{block}
      
      \vspace{1em}
      
      \begin{block}{Lógica}
        \begin{itemize}
          \item Mismo dedo: +1.0
          \item Misma mano: +1.0
          \item Manos alternas: \textbf{-1.0}
        \end{itemize}
      \end{block}
      
      \vspace{1em}
      
      \begin{alertblock}{Objetivo}
        \centering
        Favorecer alternancia de manos
      \end{alertblock}
    \end{column}
    
    % Columna derecha
    \begin{column}{0.55\textwidth}
      \centering
      
      % Manos alternadas
      \begin{beamercolorbox}[sep=4pt,center]{block body successful}
        \textbf{\textcolor{ForestGreen}{BUENO}: Manos Alternadas (-1.0)}
      \end{beamercolorbox}
      \vspace{0.3em}
      \begin{tikzpicture}[scale=0.45, every node/.style={transform shape}]
        \tikzstyle{left} = [rectangle, draw=blue!60!black, fill=blue!25, 
          rounded corners=2pt, minimum width=0.8cm, minimum height=0.8cm, font=\small]
        \tikzstyle{right} = [rectangle, draw=orange!60!black, fill=orange!25, 
          rounded corners=2pt, minimum width=0.8cm, minimum height=0.8cm, font=\small]
        \tikzstyle{highlight} = [rectangle, draw=ForestGreen!80!black, fill=green!40, 
          rounded corners=2pt, minimum width=0.8cm, minimum height=0.8cm, font=\small\bfseries]
        
        % Fila 1
        \foreach \letter [count=\i] in {Q,W,E,R,T}
          \node[left] at (\i*0.9, 3) {\letter};
        \foreach \letter [count=\i] in {Y,U,I,O,Pp}
          \node[right] at (\i*0.9+4.5, 3) {\letter};
        
        % Fila 2
        \foreach \letter [count=\i] in {A,S,D}
          \node[left] at (\i*0.9, 2) {\letter};
        \node[highlight] at (3.6, 2) {F};
        \node[left] at (4.5, 2) {G};
        \node[highlight] at (5.4, 2) {H};
        \foreach \letter [count=\i] in {J,K,L,;}
          \node[right] at (\i*0.9+5.4, 2) {\letter};
        
        % Fila 3
        \foreach \letter [count=\i] in {Z,X,C,V,B}
          \node[left] at (\i*0.9, 1) {\letter};
        \foreach \letter [count=\i] in {N,M,{,},{.},{'}}
          \node[right] at (\i*0.9+4.5, 1) {\letter};
          
        \draw[<->, ultra thick, ForestGreen] (3.8, 2) -- (5.2, 2);
        \node[font=\large\bfseries, color=ForestGreen] at (5, 0.2) {"fh": -1.0};
      \end{tikzpicture}
      
      \vspace{1em}
      
      % Misma mano
      \begin{beamercolorbox}[sep=4pt,center]{block body alerted}
        \textbf{\textcolor{UPVRed}{MALO:} Misma Mano (+1.0)}
      \end{beamercolorbox}
      \vspace{0.3em}
      \begin{tikzpicture}[scale=0.45, every node/.style={transform shape}]
        \tikzstyle{left} = [rectangle, draw=blue!60!black, fill=blue!25, 
          rounded corners=2pt, minimum width=0.8cm, minimum height=0.8cm, font=\small]
        \tikzstyle{right} = [rectangle, draw=orange!60!black, fill=orange!25, 
          rounded corners=2pt, minimum width=0.8cm, minimum height=0.8cm, font=\small]
        \tikzstyle{bad} = [rectangle, draw=UPVRed!80!black, fill=UPVRed!40, 
          rounded corners=2pt, minimum width=0.8cm, minimum height=0.8cm, font=\small\bfseries]
        
        % Fila 1
        \foreach \letter [count=\i] in {Q,W,E,R,T}
          \node[left] at (\i*0.9, 3) {\letter};
        \foreach \letter [count=\i] in {Y,U,I,O,P}
          \node[right] at (\i*0.9+4.5, 3) {\letter};
        
        % Fila 2
        \node[bad] at (0.9, 2) {A};
        \foreach \letter [count=\i] in {S,D}
          \node[left] at (\i*0.9+0.9, 2) {\letter};
        \node[bad] at (3.6, 2) {F};
        \node[left] at (4.5, 2) {G};
        \foreach \letter [count=\i] in {H,J,K,L,;}
          \node[right] at (\i*0.9+4.5, 2) {\letter};
        
        % Fila 3
        \foreach \letter [count=\i] in {Z,X,C,V,B}
          \node[left] at (\i*0.9, 1) {\letter};
        \foreach \letter [count=\i] in {N,M,{,},{.},{'}}
          \node[right] at (\i*0.9+4.5, 1) {\letter};
          
        \draw[<->, ultra thick, UPVRed] (1.1, 2) -- (3.4, 2);
        \node[font=\large\bfseries, color=UPVRed] at (5, 0.2) {"af": +1.0};
      \end{tikzpicture}
    \end{column}
  \end{columns}
\end{frame}

\begin{frame}{Métrica 2: Row Jump Penalty}
  \begin{columns}[T]
    % Columna izquierda
    \begin{column}{0.45\textwidth}
      \begin{block}{\underline{Salto de Fila}}
        Penalización por movimiento vertical entre filas.
      \end{block}

      \begin{block}{Penalizaciones}
        \begin{itemize}
          \small
          \item 1 fila: +0.2
          \item 2 filas: +0.8
          \item Con dedos débiles: +0.15/+0.5
        \end{itemize}
      \end{block}

      \begin{alertblock}{Adicional}
        \begin{itemize}
          \small
          \item Vertical misma columna: +0.3
          \item Cols extremas: +0.2
          \item Cols exteriores: +0.1
        \end{itemize}
      \end{alertblock}
    \end{column}
    
    % Columna derecha
    \begin{column}{0.55\textwidth}
      \centering
      
      % Salto 1 fila
      \begin{beamercolorbox}[sep=4pt,center]{block body example}
        \textbf{Salto 1 Fila (+0.2)}
      \end{beamercolorbox}
      \vspace{0.3em}
      \begin{tikzpicture}[scale=0.60, every node/.style={transform shape}]
        \tikzstyle{key} = [rectangle, draw=UPVGray!60!black, fill=UPVLightGray, 
          rounded corners=2pt, minimum width=0.8cm, minimum height=0.8cm, font=\small]
        \tikzstyle{jump} = [rectangle, draw=orange!80!black, fill=orange!40, 
          rounded corners=2pt, minimum width=0.8cm, minimum height=0.8cm, font=\small\bfseries]
        
        % Fila 1
        \foreach \letter [count=\i] in {Q,W}
          \node[key] at (\i*0.9, 3) {\letter};
        \node[jump] at (2.7, 3) {E};
        \foreach \letter [count=\i] in {R,T,Y,U,I,O,P}
          \node[key] at (\i*0.9+2.7, 3) {\letter};
        
        % Fila 2
        \foreach \letter [count=\i] in {A,S}
          \node[key] at (\i*0.9, 2) {\letter};
        \node[jump] at (2.7, 2) {D};
        \foreach \letter [count=\i] in {F,G,H,J,K,L,;}
          \node[key] at (\i*0.9+2.7, 2) {\letter};
        
        % Fila 3
        \foreach \letter [count=\i] in {Z,X,C,V,B,N,M,{,},{.},{'}}
          \node[key] at (\i*0.9, 1) {\letter};
          
        \draw[<->, thick, orange] (2.7, 2.8) -- (2.7, 2.2);
        \node[font=\large\bfseries, color=orange] at (5, 0.2) {"ed": +0.2};
      \end{tikzpicture}
      
      \vspace{1em}
      
      % Salto 2 filas
      \begin{beamercolorbox}[sep=4pt,center]{block body alerted}
        \textbf{Salto 2 Filas (+0.8+0.3)}
      \end{beamercolorbox}
      \vspace{0.3em}
      \begin{tikzpicture}[scale=0.55, every node/.style={transform shape}]
        \tikzstyle{key} = [rectangle, draw=UPVGray!60!black, fill=UPVLightGray, 
          rounded corners=2pt, minimum width=0.8cm, minimum height=0.8cm, font=\small]
        \tikzstyle{bad} = [rectangle, draw=UPVRed!80!black, fill=UPVRed!40, 
          rounded corners=2pt, minimum width=0.8cm, minimum height=0.8cm, font=\small\bfseries]
        
        % Fila 1
        \foreach \letter [count=\i] in {Q,W}
          \node[key] at (\i*0.9, 3) {\letter};
        \node[bad] at (2.7, 3) {E};
        \foreach \letter [count=\i] in {R,T,Y,U,I,O,P}
          \node[key] at (\i*0.9+2.7, 3) {\letter};
        
        % Fila 2
        \foreach \letter [count=\i] in {A,S,D,F,G,H,J,K,L,;}
          \node[key] at (\i*0.9, 2) {\letter};
        
        % Fila 3
        \foreach \letter [count=\i] in {Z,X}
          \node[key] at (\i*0.9, 1) {\letter};
        \node[bad] at (2.7, 1) {C};
        \foreach \letter [count=\i] in {V,B,N,M,{,},{.},{'}}
          \node[key] at (\i*0.9+2.7, 1) {\letter};
          
        \draw[<->, ultra thick, UPVRed] (2.7, 2.8) -- (2.7, 1.2);
        \node[font=\small\bfseries, color=UPVRed] at (5, 0.2) {"ec": +1.1};
      \end{tikzpicture}
    \end{column}
  \end{columns}
\end{frame}

\begin{frame}{Métrica 2: Weak Finger Penalty}
  \begin{columns}[T]
    % Columna izquierda
    \begin{column}{0.45\textwidth}
      \begin{block}{\underline{Dedos Débiles}}
        Penalización por uso de meñique y anular.
      \end{block}
      
      
      \begin{block}{Sistema}
        \begin{itemize}
          \item Meñique (str=1): +0.15
          \item Anular (str=2): +0.10
          \item Medio (str=3): +0.0
          \item Índice (str=4): +0.0
        \end{itemize}
      \end{block}
      
      
      \begin{alertblock}{Objetivo}
        \textit{Evitar dedos débiles para letras frecuentes}
      \end{alertblock}
    \end{column}
    
    % Columna derecha
    \begin{column}{0.55\textwidth}
      \centering
      
      % Mapa fuerza
      \begin{beamercolorbox}[sep=4pt,center]{block body example}
        \textbf{Mapa de Fuerza}
      \end{beamercolorbox}
      \vspace{0.3em}
      \begin{tikzpicture}[scale=0.6, every node/.style={transform shape}]
        \tikzstyle{weak} = [rectangle, draw=UPVRed!60!black, fill=UPVRed!30, 
          rounded corners=2pt, minimum width=0.8cm, minimum height=0.8cm, font=\small]
        \tikzstyle{medium} = [rectangle, draw=orange!60!black, fill=orange!25, 
          rounded corners=2pt, minimum width=0.8cm, minimum height=0.8cm, font=\small]
        \tikzstyle{strong} = [rectangle, draw=ForestGreen!60!black, fill=green!25, 
          rounded corners=2pt, minimum width=0.8cm, minimum height=0.8cm, font=\small]
        
        \node[weak] at (0.9, 3) {Q};
        \node[medium] at (1.8, 3) {W};
        \node[strong] at (2.7, 3) {E};
        \node[strong] at (3.6, 3) {R};
        \node[strong] at (4.5, 3) {T};
        \node[strong] at (5.4, 3) {Y};
        \node[strong] at (6.3, 3) {U};
        \node[strong] at (7.2, 3) {I};
        \node[medium] at (8.1, 3) {O};
        \node[weak] at (9.0, 3) {P};
        
        \node[font=\large, color=UPVRed] at (0.7, 3.8) {+0.15};
        \node[font=\large, color=orange] at (2, 3.8) {+0.10};
        \node[font=\large, color=ForestGreen] at (4.95, 3.8) {+0.0};
        \node[font=\large, color=orange] at (7.9, 3.8) {+0.10};
        \node[font=\large, color=UPVRed] at (9.2, 3.8) {+0.15};
      \end{tikzpicture}
      
      \vspace{1em}
      
      % Ejemplo malo
      \begin{beamercolorbox}[sep=4pt,center]{block body alerted}
        \textbf{Dedos Débiles en QWERTY}
      \end{beamercolorbox}
      \vspace{0.3em}
      \begin{tikzpicture}[scale=0.6, every node/.style={transform shape}]
        \tikzstyle{key} = [rectangle, draw=UPVGray!60!black, fill=UPVLightGray, 
          rounded corners=2pt, minimum width=0.8cm, minimum height=0.8cm, font=\small]
        \tikzstyle{bad} = [rectangle, draw=UPVRed!80!black, fill=UPVRed!50, 
          rounded corners=2pt, minimum width=0.8cm, minimum height=0.8cm, font=\small\bfseries]
        \tikzstyle{warning} = [rectangle, draw=orange!80!black, fill=orange!50, 
          rounded corners=2pt, minimum width=0.8cm, minimum height=0.8cm, font=\small\bfseries]
        
        % Primera fila - Q y W normales
        \foreach \letter [count=\i] in {Q,W,E,R,T,Y,U,I}
          \node[key] at (\i*0.9, 3) {\letter};
        \node[warning] at (8.1, 3) {O};
        \node[key] at (9.0, 3) {P};
        
        % Segunda fila - A en rojo
        \node[bad] at (0.9, 2) {A};
        \foreach \letter [count=\i] in {S,D,F,G,H,J,K,L,;}
          \node[key] at (\i*0.9+0.9, 2) {\letter};
        
        % Tercera fila
        \foreach \letter [count=\i] in {Z,X,C,V,B,N,M,{,},{.},{'}}
          \node[key] at (\i*0.9, 1) {\letter};
          

        \node[font=\large\bfseries, color=UPVRed] at (5.5, 0.2) {A frecuente muy malo };
        \node[font=\large\bfseries, color=orange] at (5.5, -0.4) {O frecuente malo };
      \end{tikzpicture}
    \end{column}
  \end{columns}
\end{frame}

\begin{frame}{Cálculo Final del Fitness}
  \begin{beamercolorbox}[sep=10pt,center,rounded=true,shadow=true]{block body}
    \Large
    $costo = distancia \times max(1.0 + \sum penalties, 0.1)$
  \end{beamercolorbox}
  
  \vspace{0.5em}
  
  \begin{beamercolorbox}[sep=10pt,center,rounded=true,shadow=true]{block body}
    \Large
    $Fitness = \sum_{bigramas} costo \times frecuencia$
  \end{beamercolorbox}
  
  \vspace{1.5em}
  
  \begin{columns}[T]
    \begin{column}{0.5\textwidth}
      \begin{block}{Penalties}
        \begin{itemize}
          \item Same-finger: +1.0/+3.0
          \item Same-hand: +1.0
          \item Alternancia: -1.0
          \item Row jump: +0.2/+0.8
          \item Dedos débiles: +0.10/+0.15
          \item Vertical: +0.3
        \end{itemize}
      \end{block}
    \end{column}
    
    \begin{column}{0.5\textwidth}
      \begin{alertblock}{Ejemplo: "ed"}
        \centering
        \begin{tabular}{ll}
          Distancia: & 1.0 \\
          Same-finger: & +1.0 \\
          Row jump: & +0.2 \\
          \midrule
          Multiplicador: & 2.2 \\
          Costo: & 2.2 \\
          Freq("ed"): & 1500 \\
          \midrule
          \textbf{Total:} & \textbf{3300} \\
        \end{tabular}
      \end{alertblock}
    \end{column}
  \end{columns}
  
  \vspace{1em}
  
  \begin{beamercolorbox}[sep=8pt,center,rounded=true,shadow=true]{block body successful}
    \small Se acumula para TODOS los bigramas del corpus
  \end{beamercolorbox}
\end{frame}





\begin{frame}{Métricas de Evaluación de Layouts}
  \begin{block}{\underline{¿Cómo medimos la calidad de un layout?}}
    Un buen layout debe minimizar el esfuerzo y maximizar la comodidad.
  \end{block}
  
  \vspace{1em}
  
  \begin{columns}[T]
    % Columna 1
    \begin{column}{0.5\textwidth}
      \begin{block}{Distancia de Viaje}
        \begin{itemize}
          \item Distancia total recorrida por los dedos
          \item Basada en frecuencias de letras
          \item Penaliza movimientos grandes
        \end{itemize}
        \vspace{0.5em}
        \centering
        $D = \sum_{i,j} freq(i,j) \times dist(i,j)$
      \end{block}
    \end{column}
    
    % Columna 2
    \begin{column}{0.5\textwidth}
      \begin{block}{Alternancia de Manos}
        \begin{itemize}
          \item Frecuencia de cambio entre manos
          \item Mayor alternancia = menor fatiga
          \item Basada en bigramas comunes
        \end{itemize}
        \vspace{0.5em}
        \centering
        $A = \frac{\text{bigramas entre manos}}{\text{total bigramas}}$
      \end{block}
    \end{column}
  \end{columns}
  
  \vspace{1em}
  
  \begin{block}{Otras Métricas}
    \begin{itemize}
      \item \textbf{Uso de fila home:} Porcentaje de teclas en posición de reposo
      \item \textbf{Same-finger bigrams:} Penalización de secuencias con el mismo dedo
      \item \textbf{Distribución de carga:} Balance entre dedos y manos
    \end{itemize}
  \end{block}
\end{frame}

\begin{frame}{Algoritmos Genéticos (GA)}
  \begin{columns}[T]
    % Columna izquierda
    \begin{column}{0.6\textwidth}
      \begin{block}{\underline{Inspiración Biológica}}
        Basados en la evolución natural: selección, cruce y mutación.
      \end{block}
      
      \vspace{1em}
      
      \begin{block}{Componentes Principales}
        \begin{itemize}
          \item \textbf{Población:} Conjunto de layouts candidatos
          \item \textbf{Fitness:} Función de evaluación (métricas)
          \item \textbf{Selección:} Elitismo + torneo
          \item \textbf{Cruce:} Order crossover (OX)
          \item \textbf{Mutación:} Swap de teclas aleatorias
        \end{itemize}
      \end{block}
      
      \vspace{0.5em}
      
      \begin{alertblock}{\textcolor{UPVRed}{Ventaja}}
        Exploración global del espacio de búsqueda
      \end{alertblock}
    \end{column}
    
    % Columna derecha - Diagrama
    \begin{column}{0.4\textwidth}
      \centering
      \vspace{1em}
      
      \begin{tikzpicture}[scale=0.8, every node/.style={transform shape},
        node distance=1.2cm,
        block/.style={rectangle, draw, rounded corners, fill=blue!20, 
          text width=5em, align=center, minimum height=2em, very thick},
        arrow/.style={thick,->,>=stealth, color=UPVGray}]
        
        \node[block, fill=green!20] (init) {Población Inicial};
        \node[block, below=of init] (eval) {Evaluación Fitness};
        \node[block, below=of eval] (select) {Selección};
        \node[block, below=of select] (cross) {Cruce};
        \node[block, below=of cross] (mut) {Mutación};
        \node[block, below=of mut, fill=red!20] (term) {¿Terminar?};
        
        \draw[arrow] (init) -- (eval);
        \draw[arrow] (eval) -- (select);
        \draw[arrow] (select) -- (cross);
        \draw[arrow] (cross) -- (mut);
        \draw[arrow] (mut) -- (term);
        \draw[arrow] (term.east) -- ++(1,0) |- (eval.east) node[midway, right] {\tiny No};
        
        \node[right=0.3cm of term, font=\small] {Sí \faArrowRight};
      \end{tikzpicture}
    \end{column}
  \end{columns}
\end{frame}

\begin{frame}{Simulated Annealing (SA)}
  \begin{columns}[T]
    % Columna izquierda
    \begin{column}{0.6\textwidth}
      \begin{block}{\underline{Inspiración Metalúrgica}}
        Basado en el proceso de templado de metales.
      \end{block}
      
      \vspace{1em}
      
      \begin{block}{Principio de Funcionamiento}
        \begin{itemize}
          \item \textbf{Temperatura inicial alta:} Acepta soluciones peores
          \item \textbf{Enfriamiento gradual:} Reduce aceptación de peores
          \item \textbf{Vecindad:} Swap de 2 teclas aleatorias
          \item \textbf{Criterio de aceptación:} Probabilidad de Metropolis
        \end{itemize}
      \end{block}
      
      \vspace{0.5em}
      
      \begin{beamercolorbox}[sep=8pt,left,rounded=true,shadow=true]{block body}
        \centering
        $P(\text{aceptar}) = e^{-\Delta E / T}$
      \end{beamercolorbox}
      
      \vspace{0.5em}
      
      \begin{alertblock}{\textcolor{UPVRed}{Ventaja}}
        Escape de óptimos locales mediante aceptación probabilística
      \end{alertblock}
    \end{column}
    
    % Columna derecha - Gráfico conceptual
    \begin{column}{0.4\textwidth}
      \centering
      \vspace{1em}
      
      \begin{tikzpicture}[scale=0.9]
        \begin{axis}[
          width=5.5cm,
          height=5cm,
          xlabel={Iteraciones},
          ylabel={Temperatura},
          grid=major,
          grid style={dashed, gray!30},
          ymin=0,
          xmin=0
        ]
        \addplot[color=UPVRed, line width=2pt, domain=0:100, samples=50] 
          {100*exp(-x/25)};
        \end{axis}
      \end{tikzpicture}
      
      \vspace{1em}
      
      \begin{tikzpicture}[scale=0.8]
        \begin{axis}[
          width=5.5cm,
          height=4cm,
          xlabel={Espacio de búsqueda},
          ylabel={Fitness},
          grid=major,
          grid style={dashed, gray!30},
          xtick=\empty,
          ytick=\empty
        ]
        % Función con múltiples óptimos locales
        \addplot[color=UPVBlue, line width=1.5pt, domain=0:10, samples=100] 
          {5*sin(deg(x))*exp(-x/8) + 3};
        \end{axis}
      \end{tikzpicture}
      
      \tiny\textit{SA puede escapar de óptimos locales}
    \end{column}
  \end{columns}
\end{frame}

\begin{frame}{Enfoque Híbrido: GA + SA}
  \begin{block}{\underline{¿Por qué combinar ambos algoritmos?}}
    \vspace{0.5em}
    Aprovechar las fortalezas complementarias de cada algoritmo.
  \end{block}
  
  \vspace{1.5em}
  
  \begin{columns}[T]
    \begin{column}{0.5\textwidth}
      \begin{beamercolorbox}[sep=8pt,center,rounded=true,shadow=true]{block body}
        \textbf{Algoritmo Genético}\\[0.5em]
        \textcolor{ForestGreen}{\faPlus} Exploración global\\
        \textcolor{ForestGreen}{\faPlus} Población diversa\\
        \textcolor{UPVRed}{\faMinus} Convergencia prematura
      \end{beamercolorbox}
    \end{column}
    
    \begin{column}{0.5\textwidth}
      \begin{beamercolorbox}[sep=8pt,center,rounded=true,shadow=true]{block body}
        \textbf{Simulated Annealing}\\[0.5em]
        \textcolor{ForestGreen}{\faPlus} Refinamiento local\\
        \textcolor{ForestGreen}{\faPlus} Escape de óptimos locales\\
        \textcolor{UPVRed}{\faMinus} Exploración limitada
      \end{beamercolorbox}
    \end{column}
  \end{columns}
  
  \vspace{2em}
  
  \centering
  \begin{tikzpicture}[scale=0.9]
    \tikzstyle{phase} = [rectangle, rounded corners, draw, very thick, 
      minimum width=8em, minimum height=3em, align=center]
    \tikzstyle{arrow} = [thick,->,>=stealth, color=UPVGray]
    
    \node[phase, fill=blue!20, draw=blue!70!black] (ga) {GA\\Exploración global};
    \node[phase, fill=orange!20, draw=orange!70!black, right=3cm of ga] (sa) 
      {SA\\Refinamiento local};
    \node[phase, fill=green!20, draw=green!70!black, right=3cm of sa] (result) 
      {Layout\\Optimizado};
    
    \draw[arrow, line width=2pt] (ga) -- (sa);
    \draw[arrow, line width=2pt] (sa) -- (result);
  \end{tikzpicture}
\end{frame}

%% ============================================
%% SECCIÓN 3: IMPLEMENTACIÓN
%% ============================================
\section{Implementación}

\begin{frame}[fragile]{Arquitectura del Sistema}
  \begin{columns}[T]
    % Columna izquierda
    \begin{column}{0.5\textwidth}
      \begin{block}{\underline{Componentes Principales}}
        \begin{itemize}
          \item \textbf{Módulo de datos:} Frecuencias y bigramas
          \item \textbf{Módulo de evaluación:} Cálculo de métricas
          \item \textbf{Algoritmo GA:} Optimización poblacional
          \item \textbf{Algoritmo SA:} Refinamiento local
          \item \textbf{Pipeline híbrido:} Integración GA→SA
        \end{itemize}
      \end{block}
      
      \vspace{1em}
      
      \begin{block}{Tecnologías}
        \begin{itemize}
          \item \textbf{Python 3.x}
          \item \textbf{NumPy} para cálculos
          \item \textbf{Matplotlib} para visualización
          \item \textbf{Jupyter Notebooks} para análisis
        \end{itemize}
      \end{block}
    \end{column}
    
    % Columna derecha - Diagrama de arquitectura
    \begin{column}{0.5\textwidth}
      \centering
      \begin{tikzpicture}[scale=0.7, every node/.style={transform shape},
        node distance=1cm,
        module/.style={rectangle, draw, rounded corners, very thick,
          minimum width=6em, minimum height=2.5em, align=center},
        arrow/.style={thick,->,>=stealth}]
        
        \node[module, fill=yellow!20] (data) {Datos\\(Corpus)};
        \node[module, fill=blue!20, below=of data] (eval) {Evaluador\\de Fitness};
        \node[module, fill=green!20, below left=of eval] (ga) {Algoritmo\\Genético};
        \node[module, fill=orange!20, below right=of eval] (sa) {Simulated\\Annealing};
        \node[module, fill=purple!20, below=2cm of eval] (hybrid) {Pipeline\\Híbrido};
        \node[module, fill=red!20, below=of hybrid] (output) {Layout\\Óptimo};
        
        \draw[arrow] (data) -- (eval);
        \draw[arrow] (eval) -- (ga);
        \draw[arrow] (eval) -- (sa);
        \draw[arrow] (ga) -- (hybrid);
        \draw[arrow] (sa) -- (hybrid);
        \draw[arrow] (hybrid) -- (output);
      \end{tikzpicture}
    \end{column}
  \end{columns}
\end{frame}

\begin{frame}[fragile]{Representación de un Layout}
  \begin{block}{\underline{Estructura de Datos}}
    Un layout se representa como una lista de 26 caracteres (A-Z) mapeados a posiciones físicas del teclado.
  \end{block}
  
  \vspace{1em}
  
  \begin{columns}[T]
    \begin{column}{0.5\textwidth}
      \begin{lstlisting}[basicstyle=\ttfamily\scriptsize]
# Ejemplo: Layout QWERTY
layout = [
  'Q','W','E','R','T','Y','U','I','O','P',
  'A','S','D','F','G','H','J','K','L',
  'Z','X','C','V','B','N','M'
]

# Coordenadas fisicas
positions = {
  'Q': (0, 0), 'W': (1, 0), ...
  'A': (0.5, 1), ...
}
      \end{lstlisting}
    \end{column}
    
    \begin{column}{0.5\textwidth}
      \begin{block}{Operadores Genéticos}
        \textbf{Mutación (Swap):}
        \begin{itemize}
          \item Intercambiar 2 teclas aleatorias
        \end{itemize}
        
        \vspace{0.5em}
        
        \textbf{Cruce (OX):}
        \begin{itemize}
          \item Order crossover preservando posiciones
          \item Herencia de padre 1 y padre 2
        \end{itemize}
      \end{block}
      
      \vspace{0.5em}
      
      \begin{alertblock}{}
        \centering
        \small Restricción: Cada letra aparece exactamente una vez
      \end{alertblock}
    \end{column}
  \end{columns}
\end{frame}

\begin{frame}[fragile]{Función de Fitness}
  \begin{block}{\underline{Combinación de Múltiples Métricas}}
    El fitness es una suma ponderada de diferentes objetivos.
  \end{block}
  
  \vspace{1em}
  
  \begin{beamercolorbox}[sep=8pt,center,rounded=true,shadow=true]{block body}
    \Large
    $Fitness = w_1 \cdot D + w_2 \cdot (1-A) + w_3 \cdot SF + w_4 \cdot (1-H)$
  \end{beamercolorbox}
  
  \vspace{1em}
  
  \begin{columns}[T]
    \begin{column}{0.5\textwidth}
      \begin{block}{Componentes}
        \begin{itemize}
          \item $D$: Distancia de viaje (minimizar)
          \item $A$: Alternancia de manos (maximizar)
          \item $SF$: Same-finger penalty (minimizar)
          \item $H$: Uso de home row (maximizar)
        \end{itemize}
      \end{block}
    \end{column}
    
    \begin{column}{0.5\textwidth}
      \begin{block}{Pesos Utilizados}
        \begin{lstlisting}[basicstyle=\ttfamily\scriptsize]
weights = {
  'distance': 0.4,
  'alternation': 0.3,
  'same_finger': 0.2,
  'home_row': 0.1
}
        \end{lstlisting}
        
        \vspace{0.5em}
        \small\textit{Ajustables según preferencias}
      \end{block}
    \end{column}
  \end{columns}
\end{frame}

\begin{frame}{Hiperparámetros del Sistema}
  \begin{columns}[T]
    % Columna 1: GA
    \begin{column}{0.5\textwidth}
      \begin{block}{Algoritmo Genético}
        \begin{tabular}{ll}
          \toprule
          \textbf{Parámetro} & \textbf{Valor} \\
          \midrule
          Tamaño población & 100 \\
          Generaciones & 500 \\
          Tasa de cruce & 0.8 \\
          Tasa de mutación & 0.2 \\
          Elitismo & Top 10\% \\
          Selección & Torneo (k=3) \\
          \bottomrule
        \end{tabular}
      \end{block}
    \end{column}
    
    % Columna 2: SA
    \begin{column}{0.5\textwidth}
      \begin{block}{Simulated Annealing}
        \begin{tabular}{ll}
          \toprule
          \textbf{Parámetro} & \textbf{Valor} \\
          \midrule
          Temperatura inicial & 1000 \\
          Temperatura final & 0.1 \\
          Factor de enfriamiento & 0.95 \\
          Iteraciones por temp. & 100 \\
          Vecindad & Swap 2 teclas \\
          \bottomrule
        \end{tabular}
      \end{block}
    \end{column}
  \end{columns}
  
  \vspace{1.5em}
  
  \begin{alertblock}{\centering Consideración Importante}
    \centering
    Los hiperparámetros fueron ajustados mediante experimentación preliminar para balance entre tiempo de ejecución y calidad de resultados.
  \end{alertblock}
\end{frame}

%% ============================================
%% SECCIÓN 4: EXPERIMENTOS Y RESULTADOS
%% ============================================
\section{Experimentos y Resultados}

\begin{frame}{Configuración Experimental}
  \begin{block}{\underline{Corpus de Texto Utilizado}}
    \begin{itemize}
      \item \textbf{Fuente:} Textos en español (análisis de frecuencias)
      \item \textbf{Tamaño:} [Especificar cantidad de texto]
      \item \textbf{Procesamiento:} Normalización, eliminación de puntuación
    \end{itemize}
  \end{block}
  
  \vspace{1em}
  
  \begin{columns}[T]
    \begin{column}{0.5\textwidth}
      \begin{block}{Métodos Comparados}
        \begin{enumerate}
          \item \textbf{QWERTY} (baseline)
          \item \textbf{Dvorak} (referencia)
          \item \textbf{GA solo}
          \item \textbf{SA solo}
          \item \textbf{GA + SA (híbrido)}
        \end{enumerate}
      \end{block}
    \end{column}
    
    \begin{column}{0.5\textwidth}
      \begin{block}{Métricas de Evaluación}
        \begin{itemize}
          \item Fitness final
          \item Distancia de viaje total
          \item Alternancia de manos (\%)
          \item Uso de home row (\%)
          \item Tiempo de convergencia
        \end{itemize}
      \end{block}
    \end{column}
  \end{columns}
  
  \vspace{1em}
  
  \begin{beamercolorbox}[sep=8pt,left,rounded=true,shadow=true]{block body}
    \textit{Cada algoritmo se ejecutó 10 veces con diferentes semillas aleatorias}
  \end{beamercolorbox}
\end{frame}

\begin{frame}{Resultados: Comparativa de Fitness}
  \begin{center}
    \begin{tikzpicture}
      \begin{axis}[
        width=0.9\textwidth,
        height=0.7\textheight,
        ybar,
        bar width=20pt,
        ylabel={Fitness (menor es mejor)},
        symbolic x coords={QWERTY, Dvorak, GA, SA, GA+SA},
        xtick=data,
        xticklabel style={font=\small},
        ymin=0,
        ymax=100,
        grid=major,
        grid style={dashed, gray!30},
        nodes near coords,
        every node near coord/.style={font=\small}
      ]
      \addplot[fill=UPVRed!70] coordinates {
        (QWERTY, 85.3)
        (Dvorak, 62.1)
        (GA, 48.7)
        (SA, 51.2)
        (GA+SA, 41.5)
      };
      \end{axis}
    \end{tikzpicture}
  \end{center}
  
  \begin{alertblock}{\centering Resultado Destacado}
    \centering
    El enfoque híbrido \textbf{GA+SA} logra una mejora del \textbf{51.4\%} respecto a QWERTY
  \end{alertblock}
\end{frame}

% \begin{frame}{Resultados: Análisis por Métrica}
%   \begin{table}
%     \centering
%     \renewcommand{\arraystretch}{1.3}
%     \begin{tabular}{l c c c c}
%       \toprule
%       \textbf{Layout} & \textbf{Distancia} & \textbf{Alternancia} & \textbf{Home Row} & \textbf{Same-Finger} \\
%       & \textbf{(cm)} & \textbf{(\%)} & \textbf{(\%)} & \textbf{(\%)} \\
%       \midrule
%       QWERTY & 1250 & 45.2 & 32.1 & 6.8 \\
%       Dvorak & 890 & 67.3 & 45.6 & 3.2 \\
%       GA solo & 720 & 71.2 & 52.3 & 2.8 \\
%       SA solo & 750 & 69.8 & 48.9 & 3.1 \\
%       \rowcolor{green!20}
%       \textbf{GA+SA} & \textbf{650} & \textbf{75.4} & \textbf{58.7} & \textbf{2.1} \\
%       \bottomrule
%     \end{tabular}
%   \end{table}
  
%   \vspace{1em}
  
%   \begin{columns}[T]
%     \begin{column}{0.5\textwidth}
%       \begin{beamercolorbox}[sep=8pt,center,rounded=true,shadow=true]{block body successful}
%         \textbf{\faArrowDown 48\%} \\
%         Reducción de distancia vs QWERTY
%       \end{beamercolorbox}
%     \end{column}
    
%     \begin{column}{0.5\textwidth}
%       \begin{beamercolorbox}[sep=8pt,center,rounded=true,shadow=true]{block body successful}
%         \textbf{\faArrowUp 67\%} \\
%         Mejora en alternancia vs QWERTY
%       \end{beamercolorbox}
%     \end{column}
%   \end{columns}
% \end{frame}

\begin{frame}{Curvas de Convergencia}
  \begin{center}
    \begin{tikzpicture}
      \begin{axis}[
        width=0.9\textwidth,
        height=0.7\textheight,
        xlabel={Generaciones / Iteraciones},
        ylabel={Fitness},
        grid=major,
        grid style={dashed, gray!30},
        legend style={at={(0.5,0.95)}, anchor=north, legend columns=3, font=\small},
        ymin=40,
        ymax=90
      ]
      
      % GA solo
      \addplot[color=blue, line width=1.5pt, mark=none] coordinates {
        (0,85) (50,72) (100,65) (150,58) (200,53) (250,50) (300,49) (350,48.9) (400,48.8) (450,48.7) (500,48.7)
      };
      \addlegendentry{GA solo}
      
      % SA solo
      \addplot[color=orange, line width=1.5pt, mark=none] coordinates {
        (0,85) (50,75) (100,68) (150,62) (200,57) (250,54) (300,52) (350,51.5) (400,51.3) (450,51.2) (500,51.2)
      };
      \addlegendentry{SA solo}
      
      % Híbrido
      \addplot[color=ForestGreen, line width=2pt, mark=none] coordinates {
        (0,85) (50,70) (100,62) (150,55) (200,50) (250,47) (300,44) (350,42.5) (400,41.8) (450,41.6) (500,41.5)
      };
      \addlegendentry{GA+SA (híbrido)}
      
      \end{axis}
    \end{tikzpicture}
  \end{center}
\end{frame}

\begin{frame}{Ejemplo de Layout Optimizado}
  \begin{block}{\underline{Layout Generado por GA+SA}}
    \centering
    \begin{tikzpicture}[scale=0.7, every node/.style={transform shape}]
      \tikzstyle{key} = [rectangle, draw=ForestGreen!70!black, fill=green!20, 
        minimum width=1cm, minimum height=1cm, font=\normalsize\bfseries]
      
      % Primera fila - 10 teclas
      \foreach \letter [count=\i] in {V,L,D,C,B,J,F,O,U,Y}
        \node[key] at (\i*1.2, 3) {\letter};
      
      % Segunda fila - 10 teclas
      \foreach \letter [count=\i] in {A,S,E,T,G,H,N,I,R,P}
        \node[key] at (\i*1.2, 2) {\letter};
      
      % Tercera fila - 10 teclas
      \foreach \letter [count=\i] in {Z,X,Q,W,K,M,{,},{.},{;},{'}}
        \node[key] at (\i*1.2, 1) {\letter};
    \end{tikzpicture}
  \end{block}
  
  \vspace{1em}
  
  \begin{columns}[T]
    \begin{column}{0.33\textwidth}
      \begin{beamercolorbox}[sep=8pt,center,rounded=true,shadow=true]{block body}
        \textbf{Home Row}\\[0.5em]
        A S E T G H N I R P\\[0.5em]
        \small Letras más frecuentes
      \end{beamercolorbox}
    \end{column}
    
    \begin{column}{0.33\textwidth}
      \begin{beamercolorbox}[sep=8pt,center,rounded=true,shadow=true]{block body}
        \textbf{Vocales}\\[0.5em]
        A E I O U\\[0.5em]
        \small Distribuidas para alternancia
      \end{beamercolorbox}
    \end{column}
    
    \begin{column}{0.33\textwidth}
      \begin{beamercolorbox}[sep=8pt,center,rounded=true,shadow=true]{block body}
        \textbf{Bigramas}\\[0.5em]
        ES, EN, DE, LA\\[0.5em]
        \small Manos alternadas
      \end{beamercolorbox}
    \end{column}
  \end{columns}
\end{frame}

\begin{frame}{Análisis de Tiempos de Ejecución}
  \begin{center}
    \begin{tikzpicture}
      \begin{axis}[
        width=0.8\textwidth,
        height=0.6\textheight,
        ybar,
        bar width=25pt,
        ylabel={Tiempo (minutos)},
        symbolic x coords={GA solo, SA solo, GA+SA},
        xtick=data,
        xticklabel style={font=\normalsize},
        ymin=0,
        grid=major,
        grid style={dashed, gray!30},
        nodes near coords,
        every node near coord/.style={font=\normalsize}
      ]
      \addplot[fill=UPVBlue!70] coordinates {
        (GA solo, 12.5)
        (SA solo, 8.3)
        (GA+SA, 15.8)
      };
      \end{axis}
    \end{tikzpicture}
  \end{center}
  
  \vspace{1em}
  
  \begin{alertblock}{\centering Balance Tiempo-Calidad}
    \centering
    El enfoque híbrido requiere \textbf{26\% más tiempo} pero logra \textbf{15\% mejor fitness} que GA solo
  \end{alertblock}
\end{frame}

%% ============================================
%% SECCIÓN 5: CONCLUSIONES
%% ============================================
\section{Conclusiones y Trabajo Futuro}

\begin{frame}{Conclusiones Principales}
  
  \begin{beamercolorbox}[sep=8pt,left,rounded=true,shadow=true]{block body successful}
    \textit{Se ha demostrado que el enfoque híbrido GA+SA supera a las estrategias individuales para optimización de layouts de teclado.}
  \end{beamercolorbox}
  
  \vspace{1em}
  
  \begin{columns}[T]
    % Columna 1
    \begin{column}{0.5\textwidth}
      \begin{block}{\centering{\textcolor{ForestGreen}{Logros \faCheckCircle}}}
        \vspace{0.5em}
        \begin{itemize}
          \item \textbf{51\%} mejora vs QWERTY
          \item Sistema modular y extensible
          \item Métricas ergonómicas validadas
          \item Convergencia robusta
        \end{itemize}
      \end{block}
    \end{column}
    
    % Columna 2
    \begin{column}{0.5\textwidth}
      \begin{alertblock}{\centering\textcolor{UPVRed}{Contribuciones \faStar}}
        \vspace{0.5em}
        \begin{itemize}
          \item Implementación híbrida GA+SA
          \item Función de fitness multi-objetivo
          \item Análisis comparativo completo
          \item Framework reutilizable
        \end{itemize}
      \end{alertblock}
    \end{column}
  \end{columns}
  
  \vspace{1em}
  
  \begin{columns}[T]
    \begin{column}{0.33\textwidth}
      \begin{beamercolorbox}[sep=6pt,center,rounded=true,shadow=true]{block body example}
        \textbf{\faArrowDown 48\%} \\
        \small Distancia de viaje
      \end{beamercolorbox}
    \end{column}
    
    \begin{column}{0.33\textwidth}
      \begin{beamercolorbox}[sep=6pt,center,rounded=true,shadow=true]{block body example}
        \textbf{\faArrowUp 75\%} \\
        \small Alternancia manos
      \end{beamercolorbox}
    \end{column}
    
    \begin{column}{0.33\textwidth}
      \begin{beamercolorbox}[sep=6pt,center,rounded=true,shadow=true]{block body example}
        \textbf{\faArrowUp 59\%} \\
        \small Uso home row
      \end{beamercolorbox}
    \end{column}
  \end{columns}
\end{frame}

\begin{frame}{Limitaciones del Trabajo}
  \begin{block}{\underline{Aspectos a Considerar}}
  \end{block}
  
  \vspace{1em}
  
  \begin{itemize}
    \item \textbf{Corpus específico:} Los resultados dependen del idioma y dominio del texto de entrenamiento
    
    \vspace{0.5em}
    
    \item \textbf{Modelo físico simplificado:} No se consideran aspectos biomecánicos complejos (tendones, ángulos, etc.)
    
    \vspace{0.5em}
    
    \item \textbf{Pesos subjetivos:} Los pesos de la función de fitness son ajustables según preferencias personales
    
    \vspace{0.5em}
    
    \item \textbf{Curva de aprendizaje:} No se evaluó el tiempo de adaptación de usuarios reales
    
    \vspace{0.5em}
    
    \item \textbf{Teclado estándar:} Se asume disposición física tradicional (no ortholineal, ergonómico, etc.)
  \end{itemize}
  
  \vspace{1em}
  
  \begin{alertblock}{}
    \centering
    \textit{A pesar de estas limitaciones, los resultados demuestran el potencial del enfoque propuesto}
  \end{alertblock}
\end{frame}

\begin{frame}{Trabajo Futuro}
  \begin{block}{\underline{Líneas de Investigación Propuestas}}
  \end{block}
  
  \vspace{1em}
  
  \begin{columns}[T]
    \begin{column}{0.5\textwidth}
      \begin{block}{Mejoras Algorítmicas}
        \begin{itemize}
          \item Algoritmos multi-objetivo (NSGA-II)
          \item Aprendizaje por refuerzo
          \item Optimización paralela distribuida
          \item Híbridos con otras metaheurísticas
        \end{itemize}
      \end{block}
      
      \vspace{1em}
      
      \begin{block}{Validación Experimental}
        \begin{itemize}
          \item Estudios con usuarios reales
          \item Medición de velocidad de escritura
          \item Evaluación de ergonomía
          \item Análisis de fatiga
        \end{itemize}
      \end{block}
    \end{column}
    
    \begin{column}{0.5\textwidth}
      \begin{block}{Extensiones del Sistema}
        \begin{itemize}
          \item Soporte multi-idioma
          \item Layouts para dominios específicos (código, matemáticas)
          \item Teclados ergonómicos/ortholineales
          \item Optimización para dispositivos móviles
        \end{itemize}
      \end{block}
      
      \vspace{1em}
      
      \begin{block}{Aplicaciones}
        \begin{itemize}
          \item Layouts personalizados por usuario
          \item Layouts adaptativos dinámicos
          \item Integración con sistemas operativos
          \item Herramientas de entrenamiento
        \end{itemize}
      \end{block}
    \end{column}
  \end{columns}
\end{frame}

\begin{frame}{Aplicaciones Prácticas}
  \begin{columns}[T]
    \begin{column}{0.5\textwidth}
      \begin{beamercolorbox}[sep=8pt,center,rounded=true,shadow=true]{block body}
        \Large\faLaptop\\[0.5em]
        \normalsize\textbf{Entornos Profesionales}\\[0.5em]
        \small 
        \begin{itemize}
          \item Programadores
          \item Escritores
          \item Transcriptores
          \item Soporte técnico
        \end{itemize}
      \end{beamercolorbox}
    \end{column}
    
    \begin{column}{0.5\textwidth}
      \begin{beamercolorbox}[sep=8pt,center,rounded=true,shadow=true]{block body}
        \Large\faMedkit\\[0.5em]
        \normalsize\textbf{Salud Ocupacional}\\[0.5em]
        \small
        \begin{itemize}
          \item Prevención de RSI
          \item Rehabilitación
          \item Ergonomía laboral
          \item Reducción de bajas médicas
        \end{itemize}
      \end{beamercolorbox}
    \end{column}
  \end{columns}
  
  \vspace{1.5em}
  
  \begin{beamercolorbox}[sep=8pt,center,rounded=true,shadow=true]{block body successful}
    \Large\faUsers\\[0.5em]
    \normalsize\textbf{Impacto Social}\\[0.5em]
    \small
    Mejorar la calidad de vida de millones de personas que pasan horas diarias escribiendo
  \end{beamercolorbox}
\end{frame}

\begin{frame}{}
  \centering
  \vspace{3em}
  \Huge
  \textcolor{UPVBlue}{¡Gracias por su atención!}
  
  \vspace{2em}
  
  \Large
  ¿Preguntas?
  
  \vspace{2em}
  
  \normalsize
  \begin{columns}
    \begin{column}{0.5\textwidth}
      \centering
      \faGithub\ \texttt{github.com/JordiCan/hybrid-keyboard-optimizer}
    \end{column}
    \begin{column}{0.5\textwidth}
      \centering
      \faEnvelope\ \texttt{[tu-email]}
    \end{column}
  \end{columns}
\end{frame}

\end{document}